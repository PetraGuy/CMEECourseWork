\documentclass[a4paper]{article}



\title{Exploring environmental factors influencing vascular plant species richness in UK Woodlands.}
\author{Petra Guy}

\begin{document}
\maketitle


\section{Information}

Internal Supervisor…
External Supervisor, Dr Simon Smart, CEH, ssma@ceh.ac.uk
Key words: species area, mixed models, biodiversity upscaling.

\section{Project Outline}

The Biodiversity Strategy 2020 sets out the UK Government’s plan to stop net loss of biodiversity by 2020. A variety of measures, such as establishing a resilient ecological network where wildlife can survive are described,~\cite{DeptEnv}. The report recognizes the need for evidenced based strategies and monitoring. Woodlands are specifically recognised as a rich habitat requiring “active management”. 
One measure we can use to help assess biodiversity is the richness of vascular plants; providing food and habitats for other organisms, their diversity is fundamental to the health of habitat. However, measuring their richness is not straightforward. Simply counting the number of different species present is not physically possible. Therefore, methods of estimating total species richness are important. 
Relationships between species richness and area have been extensively studied using a variety of approaches such as extrapolations from occupancy rank curves, diversity or relative abundance. See Bill Kunin’s review for a summary of current methods. ~\cite{kunin2017up}. Theoretical and empirical models vary in their predictions and may only be applicable within specific habitats. For example, the canonical power law relationship may be good predictors in more homogenous natural environments, but not applicable in fragmented landscapes such as the UK
Some empirical methods, such as occupancy rank curve, OCR, (see Cang Hui in supplement to Kunin’s review, above) have been shown to be very effective, even in the UK. But what they do not consider are the environmental factors that affect the richness. These factors are intrinsically involved in the occupancy rank curve, but, in OCR models, they are not specified explicitly. OCR estimations may be a useful way to estimate richness, we cannot infer from them what factors might contribute to higher or lower values of richness. We need to know these factors in order to protect or increase biodiversity, or suggest land management changes. Where authors have considered environmental factors, it has been suggested that habitat heterogeneity may play a bigger role in species richness than area, ~\cite{JBI:JBI1825, shen2009species}. 
In this project, we will explore datasets of species counts of vascular plants in 103 UK Woodlands. Counts consist of presence data for vascular plants from sixteen nested plots in each woodland, each containing five nests from 2m2 to 200m2.  The plots were randomly located in each wood. At the plot level, variables such as pH, diameter at breast height of trees have been measured. At the whole woodland level variables such as surrounding land use are recorded. The data has been collected in 1971 and 2001 as part of a long-term survey of UK woodlands, full details of the survey and methods can be found in English Nature Research Report 653, 2005.
Data exploration will initially consist of examining the relationships between habitat heterogeneity, area, and soil pH, as well as other factors using linear regressions, pairs plots and ANOVA. 
The power law relationship first described by Arrhenius is still used to successfully fit models to species area relationships ~\cite{principles}. Tjorve reports that the power law does not describe the species area curve across all scales, changing from a convex semi-log curve at fine scales (less than 1km2) to a power law at intermediate scales. The plots in the Woodland Survey are all at fine scales, being less than 200m2. We will therefore look at the relationship between species and area alone for all the plots and see to what extent they reflect a power law or semi-log relationship. 
A model for species richness as a function of the above terms will be derived using a mixed effect model and used to predict total species richness for each woodland.  The relationship between species and area found in the preliminary examination of the woodland plots will be trialled as a term in this model. Predictions will be tested against verified, recent biological records from the NBN Gateway as well as vice county recorder’s records. Comparison of derived species richness from our model will be compared with those derived using other successful techniques such as zeta diversity, ~\cite{hui2014zeta}. Total species richness for entire woodlands will then also be explored against landscape variables such as surrounding land use, woodland circumference and accessibility.

\bibliographystyle{plain}
  \bibliography{ProposalBib}


\end{document}
\grid
\grid
